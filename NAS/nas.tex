\documentclass[a4paper,10pt]{jsarticle}
\usepackage[top=0truemm,bottom=15truemm,left=10truemm,right=10truemm]{geometry}
\usepackage[dvipdfmx]{graphicx}
\usepackage{here}
\usepackage{mathcomp}
\usepackage{color}
\usepackage{bbm}
\usepackage{comment}
\usepackage{url}
\usepackage{listings,jvlisting} % 日本語のコメントアウトをする場合には「jvlisting(もしくはjlisting)」が必要

%ここからソースコードの表示に関する設定
\lstset{
  basicstyle={\ttfamily},
  identifierstyle={\small},
  commentstyle={\smallitshape},
  keywordstyle={\small\bfseries},
  ndkeywordstyle={\small},
  stringstyle={\small\ttfamily},
  frame={tb},
  breaklines=true,
  columns=[l]{fullflexible},
  numbers=left,
  xrightmargin=0zw,
  xleftmargin=3zw,
  numberstyle={\scriptsize},
  stepnumber=1,
  numbersep=1zw,
  lineskip=-0.5ex
}

%ここまでソースコードの表示に関する設定


%図と図の間の余白
\setlength\floatsep{0pt}

%本文と図の間の余白
\setlength\textfloatsep{0pt}

%本文中の図の余白
\setlength\intextsep{0pt}

%図とキャプションの間の余白
\setlength\abovecaptionskip{0pt}

% ページ番号ってどうやってつけるん?
%\pagestyle{plain}

\title{NAS}
\author{BlackNihonkai}
\date{更新日:\today}

\renewcommand{\lstlistingname}{コード}
\begin{document}
% 設定したタイトルなどを表示
% 『タイトル - title, author, date, maketitle』|https://medemanabu.net/latex/title-author-date-maketitle/
\maketitle

% 本文書き始め

% コメントアウト
% 文字サイズの指定:{\large 大きい文字}
% 番号付きの箇条書き
%\begin{enumerate}
%  \item 項目1
%  \item 項目2
%\end{enumerate}}
% 『Tex入門』|https://www.comp.tmu.ac.jp/tsakai/lectures/intro_tex.html

\section{sambaのインストール}
    「samba」を使うことでファイルサーバを実装する。
    apt形式でのインストールの場合は\ref{smb-apt}に示す。
    インストール後にsamba上でログインするためのユーザを\ref{smb-user}に従って登録する。
    samba用のユーザは必須で複数人でファイルサーバを使う場合にはそれぞれにユーザが必要になる。

    \begin{lstlisting}[caption=sambaのインストール,label=smb-apt]
        sudo apt install samba
    \end{lstlisting}

    \begin{lstlisting}[caption=sambaのユーザ登録,label=smb-user]
        sudo pdbedit -a ユーザ名
    \end{lstlisting}

\section{sambaの設定}
    \subsection{smb.confの編集}
        sambaの設定ファイルは「/etc/samba/」に生成される。
        「smb.conf」のバックアップをとってから編集する。
        「smb.conf」ファイルの最後尾に\ref{smb-conf}に示されるものを追加する
        \ref{smb-conf}中の「[share]」にかかれているものがサーバ名として表示される。

        \begin{lstlisting}[caption=smb.confの追加内容,label=smb-conf]
            [share]
            path = /path/to/directory
            writeable = yes
            create mode = 0777
            directory mode = 0777
        \end{lstlisting}
        
    
    \subsection{構文チェック}
        以下に示すコマンドで構文チェックをしておく。
        「Loaded services file OK.」と出力されれば問題ない。

        \begin{lstlisting}[caption=sambaの構文チェック,label=smb-check]
            testparm
        \end{lstlisting}

\section{sambaの実行}
    sambaサーバを実行する。
    \ref{smb-restart}のようにsmb.confファイルの設定を反映して実行する。
    \ref{smb-enable}のようにすることで、サーバ機の起動時にsambaサーバが自動で実行されるようになる。

    \begin{lstlisting}[caption=設定の反映とsambaサーバの実行,label=smb-restart]
        sudo systemctl restart smbd nmbd
    \end{lstlisting}

    \begin{lstlisting}[caption=サーバの自動起動,label=smb-enable]
        sudo systemctl enable smbd nmbd
    \end{lstlisting}
\end{document}
